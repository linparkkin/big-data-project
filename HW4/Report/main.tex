\documentclass[a4paper, 11pt]{article}
\usepackage{comment} % enables the use of multi-line comments (\ifx \fi) 
\usepackage{lipsum} %This package just generates Lorem Ipsum filler text. 
\usepackage{fullpage} % changes the margin

\begin{document}
%Header-Make sure you update this information!!!!
\noindent
\large\textbf{Homework 4} \hfill \textbf{Group 18} \\
\normalsize Department of Information Engineering \hfill Aladdine Ayadi\\
Big Data Computing \hfill Giovanni Barbieri\\
Prof. Andrea Pietracaprina \hfill Alessandro Pelizzo\\
Due Date: 31/05/2018 \hfill Davide Talon


\section*{Test results}
The present report shows the obtained results for the $4-th$ homework which has the goal to study the performances of a cluster in the Diversity Maximization problems. In particular the our experiment focus on the

Let $N_{exec}$ be the number of cores per executor for a total number of $N_{tot}$  cores used. The measurements evaluate the time $t_c$ which denotes the time for the construction of the coreset, $t_s$ represents the time for the run of sequential algorithm and finally $t_l$ the time needed for the loading and counting of the dataset, all times are expressed in ms.


\bgroup
\def\arraystretch{1.3}%  1 is the default, change whatever you need
\begin{footnotesize}
	\begin{table}[!htb]
		\centering
		\vspace{10pt}
		\caption{Results with dataset \textit{all}, $P=32$, $k=20$  and increasing cores.}\label{table:a}
		\begin{tabular}{c|c|c|c|c} 
			\hline
			\textbf{$N_{tot}$} & \textbf{$N_{exec}$} & \textbf{$t_c [ms]$} & \textbf{$t_s [ms]$} & \textbf{$t_l [ms]$} \\
			\hline
			4 & 2 & 14785 & 165 & 169256\\ 
			8 & 2 & 7262 & 194 & 92361 \\
			8 & 4 & 12393 & 174 & 114576\\
			8 & 8 & 23183 & 324 & 229915\\
			16 & 4 & 22651 & 229 & 147105\\
			16 & 8 & 22902 & 212 & 215455\\
			32 & 4 & 3037 & 195 & 33901\\
			32 & 8 & 24426 & 516 & 227701\\
			64 & 4 & 19192 & 299 & 148242\\
			64 & 8 & 22688& 188 & 206375\\
			\hline
		\end{tabular}
	\end{table}
\end{footnotesize}
\egroup


\bgroup
\def\arraystretch{1.3}%  1 is the default, change whatever you need
\begin{footnotesize}
	\begin{table}[!htb]
		\centering
		\vspace{10pt}
		\caption{Results with dataset \textit{all}, $N_{tot} = 32$, $N_{exec} = 4$, $k=20$  and increasing $P$.}\label{table:b}
		\begin{tabular}{c|c|c|c} 
			\hline
			\textbf{$P$} & \textbf{$t_c [ms]$} & \textbf{$t_s [ms]$} & \textbf{$t_l [ms]$} \\
			\hline
			4 &  12402 &  21 & 346732\\ 
			8 & 7808 &  36  & 42784\\
			16 &  4885 &  80 & 40415\\
			32 & 3037 &  195 & 33901\\
			64 &  9715&  607& 44835\\
			128 &  4139 & 2050 & 43149\\
			256 & 5222 &  8189 & 38288 \\

			\hline
		\end{tabular}
	\end{table}
\end{footnotesize}
\egroup

\bgroup
\def\arraystretch{1.3}%  1 is the default, change whatever you need
\begin{footnotesize}
	\begin{table}[!htb]
		\centering
		\vspace{10pt}
		\caption{Results with $N_{tot} = 32$, $N_{exec} = 4$, $k=20$, constant number of partitions $P$  and increasing size of the dataset.}\label{table:c}
		\begin{tabular}{l|c|c|c} 
			\hline
			\textbf{Dataset} & \textbf{$t_c [ms]$} & \textbf{$t_s [ms]$} & \textbf{$t_l [ms]$} \\
			\hline
			vectors-50-500000 &  1848 &  170 & 26885\\
			vectors-50-1000000 &  2190  &  193  & 31254 \\ 
			vectors-50-2000000 &  5325&    179& 73585\\
			vectors-50-3000000&  2486 &  232 &40012 \\
			vectors-50-all &  3037 &  195 & 33901\\

			\hline
		\end{tabular}
	\end{table}
\end{footnotesize}
\egroup



\end{document}
